%% Introduction to latex facilities.
%% Sat 31 Dec 2005
%% Stephen Eglen.

%% Text following a percent sign (%) until the end of line is treated
%% as a comment.

\documentclass{IEEEtran}

%%%%%%%%%%%%%%%%%%%%%%%%%%%%%%%%%%%%%%%%%%%%%%%%%%%%%%%%%%%%%%%%%%%%%%
%% This section is called the preamble, where we can specify which
%% latex packages we required.  Most (but not of all) of the packages
%% below should be fairly standard in most latex documents.  The
%% exception is xspace and the new \latex command, which you probably
%% do not need.
%%%%%%%%%%%%%%%%%%%%%%%%%%%%%%%%%%%%%%%%%%%%%%%%%%%%%%%%%%%%%%%%%%%%%%

%% Better math support:
\usepackage{amsmath}

%% Bibliography style:
\usepackage{mathptmx}           % Use the Times font.
\usepackage{graphicx}           % Needed for including graphics.
\usepackage{url}                % Facility for activating URLs.

%% Set the paper size to be A4, with a 2cm margin 
%% all around the page.
\usepackage[a4paper,margin=2cm]{geometry}

%% Natbib is a popular style for formatting references.
\usepackage{natbib}
%% bibpunct sets the punctuation used for formatting citations.
\bibpunct{(}{)}{;}{a}{,}{,}

%% textcomp provides extra control sequences for accessing text symbols:
\usepackage{textcomp}
\newcommand*{\micro}{\textmu}
%% Here, we define the \micro command to print a text "mu".
%% "\newcommand" returns an error if "\micro" is already defined.

%% This is an example of a new macro that I've created to save me
%% having to type \LaTeX each time.  The xspace command provides space
%% after the word LaTeX where appropriate.
\usepackage{xspace}
\providecommand*{\latex}{\LaTeX\xspace}
%% "\providecommand" does nothing if "\latex" is already defined.


%%%%%%%%%%%%%%%%%%%%%%%%%%%%%%%%%%%%%%%%%%%%%%%%%%%%%%%%%%%%%%%%%%%%%%
%% Start of the document.
%%%%%%%%%%%%%%%%%%%%%%%%%%%%%%%%%%%%%%%%%%%%%%%%%%%%%%%%%%%%%%%%%%%%%%

\begin{document}

\author{Frederico Wieser\\
  Department of Physics and Astronomy\\
  University College London\\
\date{\today}
\title{}
\maketitle

\begin{abstract}
  The purpose of this short document is to provide a brief overview of
  the facilities that \latex offers for formatting scientific reports.
  Furthermore, the source files for regenerating this report are
  freely available so that users can easily start writing their own
  reports using \latex.
\end{abstract}

\section{Introduction}

\latex is a typesetting program; given an input file with formatting
instructions (e.g intro.tex), the program will create your document in
one of several formats (DVI, Postscript or PDF).  It is therefore not
a WYSIWYG word processor.  \latex is known as a logical markup
language, similar for example to HTML, so that you describe a piece of
text as a ``section heading'' rather than saying that it should be
formatted in a certain way.  It has excellent facilities for
typesetting mathematics, and handles large documents (such as theses)
well.  The aim of this document is not to provide an overview of
\latex, since many other guides have already been written (see
Section~\ref{sec:summary}).  Instead, it has been written primarily to
provide simple workable examples that you can cut and paste to help
you get started with \latex.  The examples have been selected to be
those most likely to be useful when writing a scientific report.  This
document is best read by comparing the source code with the resulting
output.

\section{Running \latex}

The files to accompany this paper are at:
\url{http://www.damtp.cam.ac.uk/user/eglen/texintro}.  Get the
following files and put them into a new directory.

\begin{enumerate}
\item \url{intro.tex}: the main \latex document.
\item \url{example.bib}: a short bibliography.
\item \url{sigmoid.ps}: example postscript image.
\item \url{sigmoid.pdf}: example PDF image.
\end{enumerate}

Change directory to where you stored the files and type the
following (ignoring comments placed after \#\#):

\begin{verbatim}
latex intro                     ## Run latex 1st time.
bibtex intro                    ## Extract required references
latex intro                     ## Run latex 2nd to resolve references.
latex intro                     ## Probably need to run latex a 3rd time.
xdvi intro                      ## View the DVI (device independent) file.
dvips -o intro.ps intro         ## Create a postscript file for printing.
\end{verbatim}

You will notice that you run latex several times here; this is so that
references can be resolved, and references can be extracted from your
bibtex file.  After running latex, you will be told if you need to run
it again to resolve references.  After a while, you will get the idea
of how many times you need to run latex to resolve all your
references.

If instead you would like to generate PDF files (see
Section~\ref{sec:graphics} for a discussion of file formats for
included images), you can try the following shorter sequence:

\begin{verbatim}
pdflatex intro
bibtex intro
pdflatex intro
pdflatex intro
xpdf intro.pdf                  ## View the resulting PDF
\end{verbatim}

Whether you prefer to generate DVI or PDF is up to you.  The xdvi
viewer has some nice features, such as it can reload your document
easily and has a ``magnifying glass'' that is activated by the mouse.
On the other hand, xpdf will display the document more accurately as 
it will be printed.

\section{Tables}

Tables are relatively straightforward to generate.  Note that tables
and figures are not always placed exactly where you wish, as
they can \textit{float} to other parts of the document.  Rather than
trying to battle with \latex as to where they are placed, concentrate
first on getting the right content and let \latex worry about the
positioning.  Instead, use labels to your tables to refer to them.
See Table~\ref{tab:simple} and Table~\ref{tab:pars} for examples.

\begin{table}
  \centering
  \begin{tabular}{ccc}
    year & min temp (\textdegree C) & max temp (\textdegree C)\\ 
    \hline
    1970 & $-5$ & 35\\
    1975 & $-7$ & 29\\
    1980 & $-3$ & 30\\
    1985 & $-2$ & 32\\
  \end{tabular}
  \caption{Fictional minimal and maximal temperatures recorded in
    Cambridge over several years.}
  \label{tab:simple}
\end{table}
%% Why are the negative numbers above enclosed in math mode?
%% Hint: consider the difference between "-" in text and in math.

\begin{table}[htbp]
  \centering
  \begin{tabular}{lccc}\\ \hline
              & \multicolumn{1}{c}{$\phi$ (\micro m)}
              & \multicolumn{1}{c}{$\alpha$}
              & $\delta_{12}$ (\micro m)\\ \hline
    W81S1\\
    $h_{11}(u)$  & 67.94 & 7.81\\
    $h_{22}(u)$  & 66.27 & 5.40\\
    $h_{12}(u)$  &       &     &18\\
    \hline
    M623\\
    $h_{11}(u)$  &112.79 &  3.05\\
    $h_{22}(u)$  & 65.46 &  8.11\\
    $h_{12}(u)$  &       &      &20\\
        \hline
  \end{tabular}
  \caption{Summary of parameter estimates for the univariate
    functions $h_{11}(u)$, $h_{22}(u)$ and the bivariate function
    $h_{12}(u)$.  For the univariate fits, $\alpha$ and $\phi$ are 
    least-square estimates (assuming $\delta$ was fixed at 15 \micro m).
    The final column gives the
    maximum likelihood estimate of $\delta_{12}$ assuming that the
    interaction between types is simple inhibition.
    \label{tab:pars}}
\end{table}


\section{Bibliography management}

Scientific reports normally require a section where your references
are listed.  Bibtex is an excellent system for maintaining references,
especially for large documents.  Each reference needs a unique key;
you can then refer to the reference in your \latex document by using
this key within a cite command.

Take care when formatting your references, especially when it comes to
writing authors names and the case of letters in journal titles.  In
our examples, the files are found in \url{example.bib}.  As an example
of a citation, see \citep{ihaka1996} or \citep{ihaka1996,venables1999}.

Bibtex is flexible enough to format your references in a wide number
of different styles to suit your needs.  In this file I have used the
``natbib'' package, which is suitable for the natural sciences.
Depending on the type of cite command you get (and the package that
you use for citations), you can get different styles of citation.  See
Table~\ref{tab:cite} for some examples.

\begin{table}
  \centering
  \begin{tabular}{ll}
    \hline
    command & result\\ \hline
    \verb+\citep{ihaka1996}+ & \citep{ihaka1996}\\
    \verb+\citet{ihaka1996}+ & \citet{ihaka1996}\\
    \verb+\citep[see][p. 300]{ihaka1996}+ &
    \citep[see][p. 300]{ihaka1996}
    \\
    \verb+\citeauthor{ihaka1996}+ & \citeauthor{ihaka1996}
    \\
    \verb+\citeyear{ihaka1996}+ & \citeyear{ihaka1996}
    \\
    \hline
  \end{tabular}
  \caption{Examples of different citation commands available in the
    natbib package.}
  \label{tab:cite}
\end{table}


\section{Graphics}
\label{sec:graphics}

\latex can include images in one of several format, depending on
whether you use latex (postscript format required) or pdflatex (either
jpeg, png or pdf required).  Figures can be included either at their
natural size, or you can specify e.g. the figure width.
Figure~\ref{fig:example} shows an example image which intentionally
looks slightly different depending on whether you compile the document
with latex or pdflatex.  Note that in this example the suffix of the
image file is not included so that this document compiles under both
latex and pdflatex.

\begin{figure}
  \centering
  \includegraphics[width=6cm]{sigmoid}
  \caption{Example of a sigmoidal curve generated by the R programming
    environment.  The title above the curve indicates whether you have
  included the postscript or the pdf version of the figure.}
  \label{fig:example}
\end{figure}

\section{Mathematics}

\latex can format mathematics with ease, either in line, such as 
$x \times y$, or on separate lines, such as:
\[ x^2 +y^2 = z^2 \]

If you are writing several lines of equations, you can use statements
like the following:

\begin{align}
  b(t) & = s(t) - \int_{0}^{T} a(t') \cdot i(T-t') dt'
  \\
  a(t) & = \int_{0}^{T} b(t) \cdot e(T-t') dt' \label{eq:am}
  \\
  g(t) & = b(t) \ast e(t) \nonumber
\end{align}

By using labels on certain equations, we can refer to equations by
number, such as equation~(\ref{eq:am}).

\section{Summary}
\label{sec:summary}
This short guide should give you a flavour of what can be done with
\latex.  It is by no means complete, or supposed to be
self-explanatory.  It is, however, hopefully enough to get you
started!  Try experimenting by editing the source file and then
recompiling this document.  As mentioned earlier, there are many
guides for latex.  Two that I can recommend are
\url{http://www.andy-roberts.net/misc/latex/index.html} and 
`` The (Not So) Short Introduction to LaTeX2e''
(\url{http://ctan.tug.org/tex-archive/info/lshort/english/lshort.pdf}).


%%%%%%%%%%%%%%%%%%%%%%%%%%%%%%%%%%%%%%%%%%%%%%%%%%%%%%%%%%%%%%%%%%%%%%
%% Finally we specify the format required for our references and the
%% name of the bibtex file where our references should be taken from.
%%%%%%%%%%%%%%%%%%%%%%%%%%%%%%%%%%%%%%%%%%%%%%%%%%%%%%%%%%%%%%%%%%%%%%

\bibliographystyle{plainnat}
\bibliography{example}

\end{document}

%%%%%%%%%%%%%%%%%%%%%%%%%%%%%%%%%%%%%%%%%%%%%%%%%%%%%%%%%%%%%%%%%%%%%%
%% The end.
%%%%%%%%%%%%%%%%%%%%%%%%%%%%%%%%%%%%%%%%%%%%%%%%%%%%%%%%%%%%%%%%%%%%%%